\ifthmtools
\chapter{\pkg{thmtools} test}
\else
\chapter{\pkg{theorem-keys} test}
\fi

\section{Some Theorems} \label{mysec}

\begin{theorem}[name=Euclid,restate=firsteuclid]
\label{thm:euclid}%
For every prime $p$, there is a prime $p'>p$.
In particular, the list of primes,
\begin{equation}\label{eq:1}
2,3,5,7,\dots
\end{equation}
is infinite.
\end{theorem}

\begin{theorem}
\label{thm:blub}%
Blub
\end{theorem}

\autoref{thm:euclid}

\cref{thm:euclid}

\crefrange{thm:euclid}{thm:blub}

\autoref{eq:1}

\begin{theoremS}[Euclid]
For every prime $p$, there is a prime $p'>p$.
In particular, there are infinitely many primes.
\end{theoremS}

\begin{exercise}
Prove Euclid’s Theorem.
\end{exercise}

\begin{lem}[label=gth]
For every prime $p$, there is a prime $p'>p$.
In particular, there are infinitely many primes.
\end{lem}

\autoref{gth}

\cref{gth}

\Cref{gth}

\begin{euclid}
For every prime $p$, there is a prime $p'>p$.
In particular, there are infinitely many primes.
\end{euclid}

\begin{remark}[name=AAA,label=abc]
This is a remark.
\end{remark}

\nameref{abc}

\begin{callmeal}[Simon]\label{simon}
One
\end{callmeal}
\begin{callmeal}\label{garfunkel}
and another, and together,
\autoref{simon}, \nameref{simon},
and \cref{garfunkel} are referred
to as \cref{simon,garfunkel}.
\Cref{simon,garfunkel}, if you are at
the beginning of a sentence.
\end{callmeal}

\nameref{mysec}

\begin{styledtheorem}[Euclid]
For every prime $p$\dots
\end{styledtheorem}

\ifthmtools
\firsteuclid*
\else
\firsteuclid
\fi

\begin{theorem}[name=Keyed theorem,
label=thm:key]
This is a
key-val theorem.
\end{theorem}
\begin{theorem}[continues=thm:key]
And it’s spread out.
\end{theorem}

\subsection{Theorem with no name}

\begin{noname}
\kant[2][1]
\end{noname}

\begin{noname}[heading]
\kant[2][1]
\end{noname}

\subsection{Theorem with no number}

\begin{euclid}
\kant[2][1]
\end{euclid}

\begin{euclid}[heading]
\kant[2][1]
\end{euclid}

\subsection{Theorem with no name and no number}

\begin{nonamenonumber}
\kant[2][1]
\end{nonamenonumber}

\begin{nonamenonumber}[heading]
\kant[2][1]
\end{nonamenonumber}

\chapter{Test every key}

\PrintTheorems